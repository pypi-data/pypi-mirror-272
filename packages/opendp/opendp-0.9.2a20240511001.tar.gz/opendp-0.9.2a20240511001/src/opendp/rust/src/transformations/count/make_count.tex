\documentclass{article}
% common styling and macros shared by all proof files

\usepackage[top=1in, right=1in, left=1in, bottom=1.5in]{geometry}

\usepackage{amsmath,amsthm,amsfonts,amssymb,amscd}
\usepackage{listings}
\usepackage{hyperref}
\usepackage{xcolor}
\usepackage{xr}

\usepackage{enumerate} 
\usepackage{physics}
\usepackage{fancyhdr}
\usepackage{hyperref}
\usepackage{graphicx}
\usepackage{tcolorbox}
\usepackage{catchfile}
\usepackage{pdftexcmds}
\usepackage[T1]{fontenc}

% hyperref
\hypersetup{
  colorlinks=true,
  linkcolor=blue,
  linkbordercolor={0 0 1}
}

% \contrib macro to indicate inclusion in "contrib".
\usepackage{tcolorbox}
\newtcolorbox{warn_box}{colback=red!5!white,colframe=red!75!black}
\newcommand{\contrib}{{\begin{warn_box}This proof resides in \textbf{``contrib''} because it has not completed the vetting process.\end{warn_box}}} 
\newcommand{\floatingPoint}{{\begin{warn_box}This implementation is susceptible to floating-point vulnerabilities.\end{warn_box}}} 

% asOfCommit macro to version a code dependency. Arguments:
%    #1: relative path to file you are dependent on
%    #2: commit hash it was last edited. If outdated, this should be the second hash in the footnoote. Otherwise,
%            git log -n 1 --pretty=format:%h -- path/to/file.rs
\makeatletter
\ifnum\pdf@shellescape=1
   % "private" command that builds a link to a blob
  \newcommand{\linkOpendpBlob}[3]{%
    \href{https://github.com/opendp/opendp/blob/#1/#2#3}{\path{#3} at commit #1}}

  % latex macro expansion has a separate phase for \input evaluation
  %     immediately evaluate a command to write a temp file to ./out containing the current directory
  \immediate\write18{[ ! -f out/cwd.txt ] && (mkdir -p out && git rev-parse --show-prefix | sed "s|_|\@backslashchar\@backslashchar\@backslashchar_|g" > out/cwd.txt)}
  %     ...and then retrieve the current working directory by loading the temp file
  \CatchFileDef\GitWorkingDir{out/cwd.txt}{\endlinechar=-1}

  % command for building the (up to date) or (outdated) status
  \newcommand{\fileStatus}[2]{%
  \setbox0=\hbox{\input|"git --no-pager log -n1 --pretty='\@percentchar H' #1 | grep -E '^#2.*'"\unskip}\ifdim\wd0=0pt
        (outdated\footnote{See new changes with \texttt{git diff #2..\input|"git --no-pager log -n1 --pretty='\@percentchar h' #1" \GitWorkingDir\path{#1}}})\else
        (up to date)\fi
  }

  \newcommand{\asOfCommit}[2]{%
      % permalink the target
      \linkOpendpBlob{#2}{\GitWorkingDir}{#1}
      % conditionally add (outdated) or (up to date) depending on matching commit hash
      \fileStatus{#1}{#2}%
  }
\else
  % simplified command if shell-escape not enabled
  \newcommand{\asOfCommit}[2]{#1 at commit #2 (unknown status\footnote{Shell-escape is not enabled. Enable \texttt{--shell-escape} to check if this proof is up-to-date with the code.})}
\fi
\makeatother

% \vettingPR macro to link a PR. Arguments:
%    #1: PR number
\newcommand{\vettingPR}[1]{\href{https://github.com/opendp/opendp/pull/#1}{Pull Request \##1}}

% for links to rustdoc items in OpenDP. Arguments:
%    #1: path to item, and designation as trait, struct, fn, etc.
%    #2: item name
\makeatletter
\ifnum\pdf@shellescape=1
  % latex macro expansion has a separate phase for \input evaluation
  %     immediately evaluate a command to write a temp file to ./out containing the base path
  \immediate\write18{[ ! -f out/rustdoc.txt ] && mkdir -p out && ([ -z `kpsewhich --var-value OPENDP_RUSTDOC_PORT` ] && echo "https://docs.rs/opendp/`head -n 1 \@backslashchar`git rev-parse --show-toplevel\@backslashchar`/VERSION | sed 's|.*-dev.*|latest|g'`" || echo "http://localhost:`kpsewhich --var-value OPENDP_RUSTDOC_PORT`") > out/rustdoc.txt}
  %     ...and then retrieve the base path by loading the temp file
  \CatchFileDef\OpenDPRustdocBase{out/rustdoc.txt}{\endlinechar=-1}
\else
  % if shell commands are not enabled, just claim latest
  \newcommand{\OpenDPRustdocBase}{https://docs.rs/opendp/latest}
\fi
\makeatother
\newcommand{\rustdoc}[2]{\href{\OpenDPRustdocBase/opendp/#1.#2.html}{\texttt{#2}}}

% for links to external dependencies. Arguments:
%    #1: crate name
%    #2: path to item, and designation as trait, struct, fn, etc.
%    #3: item name
\newcommand{\docsrs}[3]{\href{https://docs.rs/#1/latest/#1/#2.#3.html}{\texttt{#3}}}

% minted (pseudocode)
\definecolor{codegreen}{rgb}{0,0.6,0}
\definecolor{codegray}{rgb}{0.5,0.5,0.5}
\definecolor{codepurple}{rgb}{0.58,0,0.82}
\definecolor{backcolour}{rgb}{0.95,0.95,0.92}

\lstdefinestyle{mystyle}{
    backgroundcolor=\color{backcolour},   
    commentstyle=\color{codegreen},
    keywordstyle=\color{magenta},
    numberstyle=\tiny\color{codegray},
    stringstyle=\color{codepurple},
    basicstyle=\ttfamily\footnotesize,
    breakatwhitespace=false,         
    breaklines=true,                 
    captionpos=b,                    
    keepspaces=true,                 
    numbers=left,                    
    numbersep=5pt,                  
    showspaces=false,                
    showstringspaces=false,
    showtabs=false,                  
    tabsize=2
}

\lstset{style=mystyle}

% common commands
\theoremstyle{definition}
\newtheorem{theorem}{Theorem}[section]
\newtheorem{lemma}[theorem]{Lemma}
\newtheorem{definition}[theorem]{Definition}
\newtheorem{warning}{Warning}
\newtheorem{corollary}{Corollary}
\newtheorem{proposition}{Proposition}
\newtheorem{remark}{Remark}
\newtheorem{observation}{Observation}
\newtheorem{note}{Note}

\newcommand{\vicki}[1]{{ {\color{olive}{(vicki)~#1}}}}
\newcommand{\hanwen}[1]{{ {\color{purple}{(hanwen)~#1}}}}
\newcommand{\zach}[1]{{ {\color{red}{(zach)~#1}}}}

\newcommand{\MultiSet}{\mathrm{MultiSet}}
\newcommand{\len}{\mathrm{len}}
\newcommand{\din}{\texttt{d\_in}}
\newcommand{\dout}{\texttt{d\_out}}
\newcommand{\T}{\texttt{T} }
\newcommand{\F}{\texttt{F} }
\newcommand{\Map}{\texttt{Map}}
\newcommand{\X}{\mathcal{X}}
\newcommand{\Y}{\mathcal{Y}}
\newcommand{\True}{\texttt{True}}
\newcommand{\False}{\texttt{False}}
\newcommand{\clamp}{\texttt{clamp}}
\newcommand{\function}{\texttt{function}}
\newcommand{\float}{\texttt{float }}
\newcommand{\questionc}[1]{\textcolor{red}{\textbf{Question:} #1}}


\newcommand{\validTransformation}[2]{%
  For every setting of the input parameters #1 to #2 such that the given preconditions
  hold, #2 raises an exception (at compile time or run time) or returns a valid transformation. A valid transformation has the following properties:
  \begin{enumerate}
      \item \textup{(Appropriate output domain).} 
      For every element $v$ in \texttt{input\_domain}, $\function(v)$ is in \texttt{output\_domain} or raises a data-independent runtime exception.
      
      \item \textup{(Stability guarantee).} 
      For every pair of elements $u, v$ in \texttt{input\_domain} and for every pair $(\din, \dout)$, 
      where \din\ has the associated type for \texttt{input\_metric} and \dout\ has the associated type for \\ \texttt{output\_metric}, 
      if $u, v$ are \din-close under \texttt{input\_metric}, $\texttt{stability\_map}(\din)$ does not raise an exception,
      and $\texttt{stability\_map}(\din) \leq \dout$, 
      then $\function(u), \function(v)$ are $\dout$-close under \texttt{output\_metric}.
  \end{enumerate}
}


\newcommand{\validMeasurement}[2]{%
  For every setting of the input parameters #1 to #2 such that the given preconditions
  hold, #2 raises an exception (at compile time or run time) or returns a valid measurement. A valid measurement has the following property:
  \begin{enumerate}
      \item \textup{(Privacy guarantee).}
      For every pair of elements $u, v$ in \texttt{input\_domain} and for every pair $(\din, \dout)$,
      where \din\ has the associated type for \texttt{input\_metric} and \dout\ has the associated type for \\ \texttt{output\_measure},
      if $u, v$ are \din-close under \texttt{input\_metric}, $\texttt{privacy\_map}(\din)$ does not raise an exception,
      and $\texttt{privacy\_map}(\din) \leq \dout$,
      then $\function(u), \function(v)$ are $\dout$-close under \texttt{output\_measure}.
  \end{enumerate}
}


\title{\texttt{fn make\_count}}
\author{S\'ilvia Casacuberta, Grace Tian, Connor Wagaman}
\date{}

\begin{document}

\maketitle

\contrib
Proves soundness of \rustdoc{transformations/fn}{make\_count} in \asOfCommit{mod.rs}{f5bb719}.

\texttt{make\_count} returns a Transformation that computes a count of the number of records in a vector.
The length of the vector, of type \texttt{usize}, is exactly casted to a user specified output type \texttt{TO}.
If the length is too large to be represented exactly by \texttt{TO}, 
the cast saturates at the maximum value of type \texttt{TO}.

\subsection*{Vetting History}
\begin{itemize}
    \item \vettingPR{513}
\end{itemize}

\section{Hoare Triple}
\subsection*{Precondition}
\begin{itemize}

    \item \texttt{TIA} (atomic input type) is a type with trait \rustdoc{traits/trait}{Primitive}. \texttt{Primitive} implies \texttt{TIA} has the trait bound:
    \begin{itemize}
        \item \rustdoc{traits/trait}{CheckNull} so that \texttt{TIA} is a valid atomic type for \rustdoc{domains/struct}{AtomDomain}
    \end{itemize}

    \item \texttt{TO} (output type) is a type with trait \rustdoc{traits/trait}{Number}. \texttt{Number} further implies \texttt{TO} has the trait bounds:
    \begin{itemize}
        \item \rustdoc{traits/trait}{InfSub} so that the output domain is compatible with the output metric
        \item \texttt{CheckNull} so that \texttt{TO} is a valid atomic type for \texttt{AtomDomain}
        \item \rustdoc{traits/trait}{ExactIntCast} for casting a vector length index of type \texttt{usize} to \texttt{TO}. \texttt{ExactIntCast} further implies \texttt{TO} has the trait bound:
        \begin{itemize}
            \item \rustdoc{traits/trait}{ExactIntBounds}, which gives the \texttt{MAX\_CONSECUTIVE} value of type \texttt{TO}
        \end{itemize}
        
        \item \texttt{One} provides a way to retrieve \texttt{TO}'s representation of 1
        \item \rustdoc{traits/trait}{DistanceConstant} to satisfy the preconditions of \texttt{new\_stability\_map\_from\_constant}
    \end{itemize}
\end{itemize}

\subsection*{Pseudocode}
\begin{lstlisting}[language = Python, escapechar=|]
def make_count():
    input_domain = VectorDomain(AtomDomain(TIA))
    output_domain = AtomDomain(TO) |\label{line:output-domain}|

    def function(data: Vec[TIA]) -> TO:|\label{line:TO-output}|
        size = input_domain.size(data) |\label{line:size}|
        try: |\label{line:try-catch}|
            return TO.exact_int_cast(size) |\label{line:exact-int-cast}|
        except FailedCast:
            return TO.MAX_CONSECUTIVE |\label{line:except-return}|

    input_metric = SymmetricDistance()
    output_metric = AbsoluteDistance(TO)

    stability_map = new_stability_map_from_constant(TO.one()) |\label{line:stability-map}|

    return Transformation(
        input_domain, output_domain, function,
        input_metric, output_metric, stability_map)
\end{lstlisting}

\subsection*{Postcondition}
\validTransformation{\texttt{(TIA, TO)}}{\texttt{make\_count}}

\section{Proofs}

\begin{proof} \textbf{(Part 1 -- appropriate output domain).}
    The \texttt{output\_domain} is \texttt{AtomDomain(TO)}, so it is sufficient to show that \texttt{function} always returns non-null values of type \texttt{TO}.
    By the definition of the \texttt{ExactIntCast} trait, \texttt{TO.exact\_int\_cast} always returns a non-null value of type \texttt{TO} or raises an exception.
    If an exception is raised, the function returns \texttt{TO.MAXIMUM\_CONSECUTIVE}, which is also a non-null value of type \texttt{TO}.
    Thus, in all cases, the function (from line \ref{line:try-catch}) returns a non-null value of type \texttt{TO}.
\end{proof}

Before proceeding with proving the validity of the stability map, we provide a couple lemmas.

\begin{lemma}
    \label{dsym-sens}
    $|\function(u) - \function(v)| \leq |\texttt{len(u)} - \texttt{len(v)}|$, 
    where \texttt{len} is an alias for \\ \texttt{input\_domain.size}.
\end{lemma}

\begin{proof}
    By \rustdoc{domains/trait}{CollectionDomain}, we know \texttt{size} on line \ref{line:size} is of type \texttt{usize}, 
    so it is non-negative and integral.
    Therefore, by the definition of \texttt{ExactIntCast}, 
    the invocation of \texttt{TO.exact\_int\_cast} on line \ref{line:exact-int-cast} can only fail if the argument is greater than \texttt{TO.MAX\_CONSECUTIVE}.
    In this case, the value is replaced with \texttt{TO.MAX\_CONSECUTIVE}.
    Therefore, $\function(u) = min(\texttt{len(u)}, c)$, where $c = \texttt{TO.MAX\_CONSECUTIVE}$.
    We use this equality to prove the lemma:

    \begin{align*}
        |\function(u) - \function(v)| &= |min(\texttt{len(u)}, c) - min(\texttt{len(v)}, c)| \\
        &\leq |\texttt{len(u)} - \texttt{len(v)}| &&\text{since clamping is stable} \\
    \end{align*}
\end{proof}

\begin{lemma}
    \label{lemma:len-sum-equiv}
    For vector $v$ with each element $\ell\in v$ drawn from domain $\mathcal{X}$, $\texttt{len(v)} = \sum_{z\in\mathcal{X}} h_v(z)$.
\end{lemma}

\begin{proof}
    Every element $\ell \in v$ is drawn from domain $\mathcal{X}$, so summing over all $z\in \mathcal{X}$ will sum over every element $\ell\in x$. 
    Recall that the definition of \texttt{SymmetricDistance} states that $h_v(z)$ will return the number of occurrences of value $z$ in vector $v$.
    Therefore, $\sum_{z\in\mathcal{X}} h_v(z)$ is the sum of the number of occurrences of each unique value; 
    this is equivalent to the total number of items in the vector. 

    Since \rustdoc{domains/trait}{CollectionDomain} is implemented for \texttt{VectorDomain<AtomDomain<TIA>>}, 
    we depend on the correctness of the implementation 
    Conditioned on the correctness of the implementation of \texttt{CollectionDomain} for \texttt{VectorDomain<AtomDomain<TIA>>},
    the variable \texttt{size} is of type \texttt{usize} containing the number of elements in \texttt{arg}.
    Therefore, $\sum_{z\in\mathcal{X}} h_v(z)$ is equivalent to \texttt{size}.
\end{proof}

\begin{proof} \textbf{(Part 2 -- stability map).} 
    Take any two elements $u, v$ in the \\\texttt{input\_domain} and any pair $(\din, \dout)$, 
    where \din\ has the associated type for \texttt{input\_metric} and \dout\ has the associated type for \texttt{output\_metric}.
    Assume $u, v$ are \din-close under \texttt{input\_metric} and that $\texttt{stability\_map}(\din) \leq \dout$. 
    These assumptions are used to establish the following inequality:
    \begin{align*}
        |\function(u) - \function(v)| &\leq |\texttt{len(u)} - \texttt{len(v)}| &&\text{by }\ref{dsym-sens} \\
        &= |\sum_{z\in \mathcal{X}} h_{\texttt{u}}(z) - \sum_{z\in \mathcal{X}} h_{\texttt{v}}(z)| &&\text{by } \ref{lemma:len-sum-equiv} \\
        &= |\sum_{z\in \mathcal{X}}\left(h_{\texttt{u}}(z) - h_{\texttt{v}}(z)\right)| &&\text{by algebra} \\
        &\leq \sum_{z\in \mathcal{X}}|h_{\texttt{u}}(z) - h_{\texttt{v}}(z)| &&\text{by triangle inequality} \\
        &= d_{Sym}(u, v) &&\text{by } \rustdoc{metrics/struct}{SymmetricDistance}\\
        &\leq \din &&\text{by the first assumption} \\
        &\leq \texttt{TO.inf\_cast}(\din) &&\text{by } \rustdoc{traits/trait}{InfCast} \\
        &\leq \texttt{TO.one().inf\_mul(TO.inf\_cast(\din))} &&\text{by } \rustdoc{traits/trait}{InfMul} \\
        &=\texttt{stability\_map}(\din) &&\text{by pseudocode line } \ref{line:stability-map} \\
        &\leq \dout &&\text{by the second assumption}
    \end{align*}

    It is shown that \function(u), \function(v) are \dout-close under \texttt{output\_metric}.
\end{proof}

\end{document}
